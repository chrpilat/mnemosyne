This chapter describes the interface that {\sc Mnemosyne} assumes for
each memory IP. This can be used to create the wrappers for new
technologies or to extend the memory library with IPs that are
composed of multiple blocks.
%%
Given a memory IP with parameters {\tt width} and {\tt height}, {\sc
  Mnemosyne} assumes that its interface features the following global
signal:
\begin{itemize}
\item {\tt CLK} represents the clock signal for all interfaces.
\end{itemize}
\noindent Currently, there are no explicit signals to reset the memory IP. Hence, 
the designer must ensure that data are never read before being written.
\begin{lattention}
  Currently, {\sc Mnemosyne} does not support memory interfaces with
  independent clocks. Hence, if the memory IP has separated clocks for
  each interface, they must be connected to the same input signal.
\end{lattention}

\noindent Then, let $n$ be the index of the memory interface, {\sc
  Mnemosyne} assumes that the memory interface $n$ has the following
signals:
\begin{itemize}
\item {\tt An} (input) represents the physical address to be accessed
  inside the memory IP. This signal has a size of {\tt
    $log_2(height)$} bits;
\item {\tt CEn} (input, active high) indicates that an operation
  (either read or write) is active on the current memory IP. This
  signal has a size of one bit.
\item {\tt WEn} (input, active high) indicates that the memory IP is
  accessed for a write operation. This signal has a size of one bit.
\item {\tt Dn} (input) specifies the input data to be written (for
  write operations). This signal has a size of $width$ bits.
\item {\tt WEMn} (input) indicates the mask for write operations
  (i.e. which bits of the input data have to be effectively
  written). This signal has a size of $width$ bits.
\item {\tt Qn} (output) is the output value produced by a read
  operation. This signal has a size of $width$ bits.
\end{itemize}

\begin{lattention}
  The write mask feature (signal {\tt WEMn}) is not available in all
  technologies. For example, in FPGA, all bits of the input data are
  always written, regardless of the value of this signal, which is
  ignored.
\end{lattention}

\noindent Note that signals {\tt WEn}, {\tt WEMn}, and {\tt Dn} are
available only for write interfaces, while signal {\tt Qn} is
available only for read interfaces. All signals are available,
instead, for read/write interfaces.

\newpage

For example, a dual-port 1024$\times$32 memory IP for CMOS technology,
named {\tt SRAM\_1024x32}, has the following module interface:

\begin{myverilog}{}
module SRAM_1024x32 (CLK, CE0, A0, WE0, D0, WEM0, Q0, CE1, A1, WE1, D1, WEM1, Q1 );
  input         CLK;
  //first interface
  input         CE0;
  input  [9:0]  A0;
  input         WE0;
  input  [31:0] D0;
  input  [31:0] WEM0;
  output [31:0] Q0;
  //second interface
  input         CE1;
  input  [9:0]  A1;
  input         WE1;
  input  [31:0] D1;
  input  [31:0] WEM1;
  output [31:0] Q1;
  //definition of the macro given by the technology vendor
  //...
endmodule 
\end{myverilog}

\noindent Hence, it is possible to use the so-created wrappers to create complex
memory blocks by instantiating multiple replica of these modules and creating
the proper logic to drive each of the signals. For example, let {\tt A0} be a
2048$\times$32 array to be stored in the PLM with one memory-ready and one
memory-write interface. The correponding memory block is defined as follows:

\begin{myverilog}{}
module array_A0 (CLK, CE0, A0, WE0, D0, WEM0, Q0, CE1, A1, Q1 );
  input         CLK;
  //first interface
  input         CE0;
  input  [10:0] A0;
  input         WE0;
  input  [31:0] D0;
  input  [31:0] WEM0;
  //second interface
  input         CE1;
  input  [10:0] A1;
  output [31:0] Q1;
  //logic to drive interface signals from/to IP ports
  //...
  //definitions of the banks
  SRAM_1024x32 mem_0(.CLK(CLK), .CE0(mem_0_CE0), .A0(mem_0_A0),
      .D0(mem_0_D0), .WE0(mem_0_WE0), .WEM0(mem_0_WEM0), .Q0(mem_0_Q0),
      .CE1(mem_0_CE1), .A1(mem_0_A1), .Q1(mem_0_Q1),
      .D1(mem_0_D1), .WE1(mem_0_WE1), .WEM1(mem_0_WEM1));     
  SRAM_1024x32 mem_1(.CLK(CLK), .CE0(mem_1_CE0), .A0(mem_1_A0),
      .D0(mem_1_D0), .WE0(mem_1_WE0), .WEM0(mem_1_WEM0), .Q0(mem_1_Q0),
      .CE1(mem_1_CE1), .A1(mem_1_A1), .Q1(mem_1_Q1),
      .D1(mem_1_D1), .WE1(mem_1_WE1), .WEM1(mem_1_WEM1));
endmodule 
\end{myverilog}

\noindent Where the different signals \name{mem_0_XX} and \name{mem_1_XX} represent the
logic necessary to drive the ports of the two memory IPs, respectively. This
logic is then automatically generated by {\sc Mnemosyne} based on the memory
requirements. For example, since we must apply {\em block partitioning} to
allocate the array to these memory blocks, a memory-read operation is activated
on the second interface of the memory IP \name{mem_0} when the most significant
bit is zero. Similarly, the memory-read operation is activated on the memory
IP \name{mem_1} when the most significant bit is one. The rest of the address is
used to specify the offset to access the data inside the memory IP.

\begin{myverilog}{}
  assign mem_0_CE1 = (C1 & A1[10:10] == 0) ? 1'b1 : 1'b0;
  assign mem_0_A1 = (C1 & A1[10:10] == 0) ? A1[9:0] : {10{1'b0}};
  assign mem_1_CE1 = (C1 & A1[10:10] == 1) ? 1'b1 : 1'b0;
  assign mem_1_A1 = (C1 & A1[10:10] == 1) ? A1[9:0] : {10{1'b0}};
\end{myverilog}




