This chapter describes the interface that {\sc Mnemosyne} assumes for
the process interfaces of the accelerator logic. This can be used to
create components that can interface with the generated memory
wrappers.
%%
Given a process interface $n$ used to access a data structure of parameters {\tt width} and {\tt height}, {\sc
  Mnemosyne} assumes that the interface features the following signals:
\begin{itemize}
\item {\tt An} (output) represents the offset inside the data structure
(i.e. the logical address) to be accessed with the memory operation.  This
signal has a size of {\tt $log_2(height)$} bits;
\item {\tt CEn} (output, active high) indicates that there is an active
request on the current process interface (i.e. a memory operation is accessing
the corresponding data structure). This signal has a size of one bit.
\item {\tt WEn} (output, active high) indicates that there is an active
memory-write operation on the interface.  The input data is assumed to be valid
in this case. This signal has a size of one bit.
\item {\tt Dn} (output) specifies the data to be written (for
  write operations). This signal has a size of $width$ bits.
\item {\tt WEMn} (output) indicates the mask for write operations
  (i.e. which bits of the input data have to be effectively written). This
  signal has a size of $width$ bits.
\item {\tt Qn} (input) is the output value produced by a read
  operation. This signal has a size of $width$ bits.
\end{itemize}

\noindent Note that signals {\tt WEn}, {\tt WEMn}, and {\tt Dn} are
available only for write interfaces, while signal {\tt Qn} is
available only for read interfaces. All signals are available,
instead, for read/write interfaces.

{\sc Mnemosyne} generates the logic to translate the memory requests into
proper memory requests to the physical memory IPs necessary to store the
corresponding data structure. Specifically, it generates a PLM controller that
contains a configuration of memory IPs that can store the corresponding data
structure (more than one in case of bank reuse). This PLM controller has as many
interfaces as the ones requested by the accelerator logic. It contains
internally the logic for address translation.

For example, a component \name{accelerator_logic} that needs to access
a 2048$\times$32 array named {\tt A0} with two memory interfaces (one
write and one read interface) has the interface:

\begin{myverilog}{}
module accelerator_logic (CLK,
              //...
              A0_CE0, A0_A0, A0_WE0, A0_D0, A0_WEM0,
              A0_CE1, A0_A1, A0_Q1,
              //...
              );
  input         CLK;
  //.. other signals / interfaces
  //write interface
  output        A0_CE0;
  output [10:0] A0_A0;
  output        A0_WE0;
  output [31:0] A0_D0;
  output [31:0] A0_WEM0;
  //read interface
  output        A0_CE1;
  output [10:0] A0_A1;
  input  [31:0] A0_Q1;
  //... other signals / interfaces
  //... rest of the module
endmodule 
\end{myverilog}


\noindent This module can be simply connected to the corresponding PLM 
(see Section~\ref{ch:mem_interface} for its definition) with direct
signals as follows:

\begin{myverilog}{}
module accelerator (CLK,
              //...
              //other signal / interfaces
              //...
              );
  input         CLK;
  //.. other signals / interfaces
  //write interface signals
  wire        A0_CE0;
  wire [10:0] A0_A0;
  wire        A0_WE0;
  wire [31:0] A0_D0;
  wire [31:0] A0_WEM0;
  //read interface
  wire        A0_CE1;
  wire [10:0] A0_A1;
  wire [31:0] A0_Q1;
  //accelerator logic
  accelerator_logic acc_0(.CLK(CLK), 
      //...other signals / interfaces 
      .A0_CE0(A0_CE0), .A0(A0_A0),
      .A0_D0(A0_D0), .A0_WE0(A0_WE0), .A0_WEM0(A0_WEM0), 
      .A0_CE1(A0_CE1), .A0_A1(A0_A1), .A0_Q1(A0_Q1));     
  //PLM
  array_A0 mem_0(.CLK(CLK),
      .CE0(A0_CE0), A0(A0_A0), 
      .D0(A0_D0), .WE0(A0_WE0), .WEM0(A0_WEM0),
      .CE1(A0_CE1), A1(A0_A1), .Q1(A0_Q1));
endmodule 
\end{myverilog}
